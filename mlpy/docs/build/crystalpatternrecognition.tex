%% Generated by Sphinx.
\def\sphinxdocclass{report}
\documentclass[letterpaper,10pt,english]{sphinxmanual}
\ifdefined\pdfpxdimen
   \let\sphinxpxdimen\pdfpxdimen\else\newdimen\sphinxpxdimen
\fi \sphinxpxdimen=.75bp\relax

\PassOptionsToPackage{warn}{textcomp}
\usepackage[utf8]{inputenc}
\ifdefined\DeclareUnicodeCharacter
% support both utf8 and utf8x syntaxes
  \ifdefined\DeclareUnicodeCharacterAsOptional
    \def\sphinxDUC#1{\DeclareUnicodeCharacter{"#1}}
  \else
    \let\sphinxDUC\DeclareUnicodeCharacter
  \fi
  \sphinxDUC{00A0}{\nobreakspace}
  \sphinxDUC{2500}{\sphinxunichar{2500}}
  \sphinxDUC{2502}{\sphinxunichar{2502}}
  \sphinxDUC{2514}{\sphinxunichar{2514}}
  \sphinxDUC{251C}{\sphinxunichar{251C}}
  \sphinxDUC{2572}{\textbackslash}
\fi
\usepackage{cmap}
\usepackage[T1]{fontenc}
\usepackage{amsmath,amssymb,amstext}
\usepackage{babel}



\usepackage{times}
\expandafter\ifx\csname T@LGR\endcsname\relax
\else
% LGR was declared as font encoding
  \substitutefont{LGR}{\rmdefault}{cmr}
  \substitutefont{LGR}{\sfdefault}{cmss}
  \substitutefont{LGR}{\ttdefault}{cmtt}
\fi
\expandafter\ifx\csname T@X2\endcsname\relax
  \expandafter\ifx\csname T@T2A\endcsname\relax
  \else
  % T2A was declared as font encoding
    \substitutefont{T2A}{\rmdefault}{cmr}
    \substitutefont{T2A}{\sfdefault}{cmss}
    \substitutefont{T2A}{\ttdefault}{cmtt}
  \fi
\else
% X2 was declared as font encoding
  \substitutefont{X2}{\rmdefault}{cmr}
  \substitutefont{X2}{\sfdefault}{cmss}
  \substitutefont{X2}{\ttdefault}{cmtt}
\fi


\usepackage[Bjarne]{fncychap}
\usepackage{sphinx}

\fvset{fontsize=\small}
\usepackage{geometry}


% Include hyperref last.
\usepackage{hyperref}
% Fix anchor placement for figures with captions.
\usepackage{hypcap}% it must be loaded after hyperref.
% Set up styles of URL: it should be placed after hyperref.
\urlstyle{same}

\usepackage{sphinxmessages}
\setcounter{tocdepth}{1}



\title{Crystal Pattern Recognition}
\date{Apr 07, 2020}
\release{0.1}
\author{Michael Barbier}
\newcommand{\sphinxlogo}{\vbox{}}
\renewcommand{\releasename}{Release}
\makeindex
\begin{document}

\pagestyle{empty}
\sphinxmaketitle
\pagestyle{plain}
\sphinxtableofcontents
\pagestyle{normal}
\phantomsection\label{\detokenize{index::doc}}

\phantomsection\label{\detokenize{index:module-mlpy}}\index{module@\spxentry{module}!mlpy@\spxentry{mlpy}}\index{mlpy@\spxentry{mlpy}!module@\spxentry{module}}

\chapter{Package mlpy}
\label{\detokenize{index:package-mlpy}}
Crystal pattern recognition in images.

\phantomsection\label{\detokenize{index:module-0}}\index{module@\spxentry{module}!mlpy@\spxentry{mlpy}}\index{mlpy@\spxentry{mlpy}!module@\spxentry{module}}\phantomsection\label{\detokenize{index:module-mlpy.plots}}\index{module@\spxentry{module}!mlpy.plots@\spxentry{mlpy.plots}}\index{mlpy.plots@\spxentry{mlpy.plots}!module@\spxentry{module}}

\chapter{Module plots}
\label{\detokenize{index:module-plots}}
Module for plots, graphs, annotation of images

\phantomsection\label{\detokenize{index:module-1}}\index{module@\spxentry{module}!plots@\spxentry{plots}}\index{plots@\spxentry{plots}!module@\spxentry{module}}\index{draw\_circles() (in module mlpy.plots)@\spxentry{draw\_circles()}\spxextra{in module mlpy.plots}}

\begin{fulllineitems}
\phantomsection\label{\detokenize{index:mlpy.plots.draw_circles}}\pysiglinewithargsret{\sphinxcode{\sphinxupquote{mlpy.plots.}}\sphinxbfcode{\sphinxupquote{draw\_circles}}}{\emph{\DUrole{n}{circle\_list}}, \emph{\DUrole{n}{im}}, \emph{\DUrole{n}{radius}}}{}
Draw circles on an image with a specific radius.
\begin{quote}\begin{description}
\item[{Parameters}] \leavevmode\begin{itemize}
\item {} 
\sphinxstyleliteralstrong{\sphinxupquote{circle\_list}} \textendash{} List of {[}x,y,radius{]} of the circle

\item {} 
\sphinxstyleliteralstrong{\sphinxupquote{im}} \textendash{} The input image

\item {} 
\sphinxstyleliteralstrong{\sphinxupquote{radius}} \textendash{} The radius of the circles

\end{itemize}

\item[{Returns}] \leavevmode
The image with the circles annotated (writes on the original image)

\item[{Return type}] \leavevmode
image as uint8 rgb numpy matrix

\end{description}\end{quote}

\end{fulllineitems}

\index{plot\_lindemann\_histogram() (in module mlpy.plots)@\spxentry{plot\_lindemann\_histogram()}\spxextra{in module mlpy.plots}}

\begin{fulllineitems}
\phantomsection\label{\detokenize{index:mlpy.plots.plot_lindemann_histogram}}\pysiglinewithargsret{\sphinxcode{\sphinxupquote{mlpy.plots.}}\sphinxbfcode{\sphinxupquote{plot\_lindemann\_histogram}}}{\emph{\DUrole{n}{lindemann\_parameter\_list}}, \emph{\DUrole{n}{n\_bins}}}{}
Plots the Lindemann histogram TODO

\end{fulllineitems}

\phantomsection\label{\detokenize{index:module-mlpy.detection}}\index{module@\spxentry{module}!mlpy.detection@\spxentry{mlpy.detection}}\index{mlpy.detection@\spxentry{mlpy.detection}!module@\spxentry{module}}

\chapter{Module detection}
\label{\detokenize{index:module-detection}}
Module for particle/spot detection in an image/video

\phantomsection\label{\detokenize{index:module-2}}\index{module@\spxentry{module}!detection@\spxentry{detection}}\index{detection@\spxentry{detection}!module@\spxentry{module}}\index{detection() (in module mlpy.detection)@\spxentry{detection()}\spxextra{in module mlpy.detection}}

\begin{fulllineitems}
\phantomsection\label{\detokenize{index:mlpy.detection.detection}}\pysiglinewithargsret{\sphinxcode{\sphinxupquote{mlpy.detection.}}\sphinxbfcode{\sphinxupquote{detection}}}{\emph{\DUrole{n}{orig}}, \emph{\DUrole{n}{method}}, \emph{\DUrole{n}{saturation\_perc}}, \emph{\DUrole{n}{radius}}}{}
Detection of particles as centers and radii. Uses a specified method and does some pre\sphinxhyphen{}processing of the data.
\begin{quote}\begin{description}
\item[{Parameters}] \leavevmode\begin{itemize}
\item {} 
\sphinxstyleliteralstrong{\sphinxupquote{orig}} \textendash{} The original image (can be RGB or gray\sphinxhyphen{}valued)

\item {} 
\sphinxstyleliteralstrong{\sphinxupquote{method}} \textendash{} On of the valid methods: {[}‘CHT’, ‘Laplace’{]}

\item {} 
\sphinxstyleliteralstrong{\sphinxupquote{saturation\_perc}} \textendash{} Saturation percentage

\item {} 
\sphinxstyleliteralstrong{\sphinxupquote{radius}} \textendash{} Expected radius

\end{itemize}

\item[{Returns}] \leavevmode
{[}circle\_list, im\_gray, im\_norm, im\_blur{]}
circle\_list:

\end{description}\end{quote}

\end{fulllineitems}

\index{detection\_cht() (in module mlpy.detection)@\spxentry{detection\_cht()}\spxextra{in module mlpy.detection}}

\begin{fulllineitems}
\phantomsection\label{\detokenize{index:mlpy.detection.detection_cht}}\pysiglinewithargsret{\sphinxcode{\sphinxupquote{mlpy.detection.}}\sphinxbfcode{\sphinxupquote{detection\_cht}}}{\emph{\DUrole{n}{im}}, \emph{\DUrole{n}{radius}}}{}
Detect particles using the Circular Hough Transform (CHT)
\begin{quote}\begin{description}
\item[{Parameters}] \leavevmode\begin{itemize}
\item {} 
\sphinxstyleliteralstrong{\sphinxupquote{im}} \textendash{} The input image which should be grey\sphinxhyphen{}valued

\item {} 
\sphinxstyleliteralstrong{\sphinxupquote{radius}} \textendash{} The radius of the particles used by the CHT algorithm

\end{itemize}

\item[{Returns}] \leavevmode
{[}1,2{]}:
(1) A list with {[}x, y, radius{]} values,
(2) The smoothed image used as input to the CHT algorithm

\end{description}\end{quote}

\end{fulllineitems}

\index{detection\_laplace() (in module mlpy.detection)@\spxentry{detection\_laplace()}\spxextra{in module mlpy.detection}}

\begin{fulllineitems}
\phantomsection\label{\detokenize{index:mlpy.detection.detection_laplace}}\pysiglinewithargsret{\sphinxcode{\sphinxupquote{mlpy.detection.}}\sphinxbfcode{\sphinxupquote{detection\_laplace}}}{\emph{\DUrole{n}{im}}, \emph{\DUrole{n}{radius}}}{}~\begin{quote}\begin{description}
\item[{Parameters}] \leavevmode\begin{itemize}
\item {} 
\sphinxstyleliteralstrong{\sphinxupquote{im}} \textendash{} 

\item {} 
\sphinxstyleliteralstrong{\sphinxupquote{radius}} \textendash{} 

\end{itemize}

\item[{Returns}] \leavevmode


\end{description}\end{quote}

\end{fulllineitems}

\phantomsection\label{\detokenize{index:module-mlpy.patterns}}\index{module@\spxentry{module}!mlpy.patterns@\spxentry{mlpy.patterns}}\index{mlpy.patterns@\spxentry{mlpy.patterns}!module@\spxentry{module}}

\chapter{Module patterns}
\label{\detokenize{index:module-patterns}}
The machine learning methods to recognize various crystal structures.

\phantomsection\label{\detokenize{index:module-3}}\index{module@\spxentry{module}!patterns@\spxentry{patterns}}\index{patterns@\spxentry{patterns}!module@\spxentry{module}}\index{compute\_lindemann\_parameter() (in module mlpy.patterns)@\spxentry{compute\_lindemann\_parameter()}\spxextra{in module mlpy.patterns}}

\begin{fulllineitems}
\phantomsection\label{\detokenize{index:mlpy.patterns.compute_lindemann_parameter}}\pysiglinewithargsret{\sphinxcode{\sphinxupquote{mlpy.patterns.}}\sphinxbfcode{\sphinxupquote{compute\_lindemann\_parameter}}}{\emph{\DUrole{n}{circle\_list}}, \emph{\DUrole{n}{radius}}}{}
Draw circles on an image with a specific radius.
\begin{quote}\begin{description}
\item[{Parameters}] \leavevmode\begin{itemize}
\item {} 
\sphinxstyleliteralstrong{\sphinxupquote{circle\_list}} \textendash{} list of {[}x, y, radius{]}, decribing the circles

\item {} 
\sphinxstyleliteralstrong{\sphinxupquote{radius}} \textendash{} The radius describing the size of the local region around the point/particle of interest

\end{itemize}

\item[{Returns}] \leavevmode
None

\end{description}\end{quote}

\end{fulllineitems}

\phantomsection\label{\detokenize{index:module-mlpy.mio}}\index{module@\spxentry{module}!mlpy.mio@\spxentry{mlpy.mio}}\index{mlpy.mio@\spxentry{mlpy.mio}!module@\spxentry{module}}

\chapter{Module mio}
\label{\detokenize{index:module-mio}}
Module for input/output of images, text files, etc

\phantomsection\label{\detokenize{index:module-4}}\index{module@\spxentry{module}!mio@\spxentry{mio}}\index{mio@\spxentry{mio}!module@\spxentry{module}}\index{get\_metadata() (in module mlpy.mio)@\spxentry{get\_metadata()}\spxextra{in module mlpy.mio}}

\begin{fulllineitems}
\phantomsection\label{\detokenize{index:mlpy.mio.get_metadata}}\pysiglinewithargsret{\sphinxcode{\sphinxupquote{mlpy.mio.}}\sphinxbfcode{\sphinxupquote{get\_metadata}}}{\emph{\DUrole{n}{file\_path}}}{}
Extracts the meta data of an image/video using ffmpeg and puts it into a specific format (dictionary)
\begin{quote}\begin{description}
\item[{Parameters}] \leavevmode
\sphinxstyleliteralstrong{\sphinxupquote{file\_path}} \textendash{} File path of the image/video

\item[{Returns}] \leavevmode
The meta data as a dictionary

\end{description}\end{quote}

\end{fulllineitems}

\index{print\_metadata() (in module mlpy.mio)@\spxentry{print\_metadata()}\spxextra{in module mlpy.mio}}

\begin{fulllineitems}
\phantomsection\label{\detokenize{index:mlpy.mio.print_metadata}}\pysiglinewithargsret{\sphinxcode{\sphinxupquote{mlpy.mio.}}\sphinxbfcode{\sphinxupquote{print\_metadata}}}{\emph{\DUrole{n}{file\_path}}}{}
Prints the meta data as extracted by ffmpeg and returns this raw meta data.
\begin{quote}\begin{description}
\item[{Parameters}] \leavevmode
\sphinxstyleliteralstrong{\sphinxupquote{file\_path}} \textendash{} File path of the image/video

\item[{Returns}] \leavevmode
The meta data as extracted by ffmpeg

\end{description}\end{quote}

\end{fulllineitems}

\index{read\_frames() (in module mlpy.mio)@\spxentry{read\_frames()}\spxextra{in module mlpy.mio}}

\begin{fulllineitems}
\phantomsection\label{\detokenize{index:mlpy.mio.read_frames}}\pysiglinewithargsret{\sphinxcode{\sphinxupquote{mlpy.mio.}}\sphinxbfcode{\sphinxupquote{read\_frames}}}{\emph{\DUrole{n}{file\_path}}, \emph{\DUrole{n}{frame\_list}}}{}
Loads video data using the OpenCV library (reads in a specified list of frames).
\begin{quote}\begin{description}
\item[{Parameters}] \leavevmode\begin{itemize}
\item {} 
\sphinxstyleliteralstrong{\sphinxupquote{file\_path}} \textendash{} File path of the video

\item {} 
\sphinxstyleliteralstrong{\sphinxupquote{frame\_list}} \textendash{} List of frames (the indices) of interest

\end{itemize}

\item[{Returns}] \leavevmode
A list with the frames as numpy? data

\end{description}\end{quote}

\end{fulllineitems}



\chapter{Indices and tables}
\label{\detokenize{index:indices-and-tables}}\begin{itemize}
\item {} 
\DUrole{xref,std,std-ref}{genindex}

\item {} 
\DUrole{xref,std,std-ref}{modindex}

\item {} 
\DUrole{xref,std,std-ref}{search}

\end{itemize}


\renewcommand{\indexname}{Python Module Index}
\begin{sphinxtheindex}
\let\bigletter\sphinxstyleindexlettergroup
\bigletter{d}
\item\relax\sphinxstyleindexentry{detection}\sphinxstyleindexpageref{index:\detokenize{module-2}}
\indexspace
\bigletter{m}
\item\relax\sphinxstyleindexentry{mio}\sphinxstyleindexpageref{index:\detokenize{module-4}}
\item\relax\sphinxstyleindexentry{mlpy}\sphinxstyleindexpageref{index:\detokenize{module-0}}
\item\relax\sphinxstyleindexentry{mlpy.detection}\sphinxstyleindexpageref{index:\detokenize{module-mlpy.detection}}
\item\relax\sphinxstyleindexentry{mlpy.mio}\sphinxstyleindexpageref{index:\detokenize{module-mlpy.mio}}
\item\relax\sphinxstyleindexentry{mlpy.patterns}\sphinxstyleindexpageref{index:\detokenize{module-mlpy.patterns}}
\item\relax\sphinxstyleindexentry{mlpy.plots}\sphinxstyleindexpageref{index:\detokenize{module-mlpy.plots}}
\indexspace
\bigletter{p}
\item\relax\sphinxstyleindexentry{patterns}\sphinxstyleindexpageref{index:\detokenize{module-3}}
\item\relax\sphinxstyleindexentry{plots}\sphinxstyleindexpageref{index:\detokenize{module-1}}
\end{sphinxtheindex}

\renewcommand{\indexname}{Index}
\printindex
\end{document}